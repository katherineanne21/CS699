% Options for packages loaded elsewhere
\PassOptionsToPackage{unicode}{hyperref}
\PassOptionsToPackage{hyphens}{url}
%
\documentclass[
]{article}
\usepackage{amsmath,amssymb}
\usepackage{iftex}
\ifPDFTeX
  \usepackage[T1]{fontenc}
  \usepackage[utf8]{inputenc}
  \usepackage{textcomp} % provide euro and other symbols
\else % if luatex or xetex
  \usepackage{unicode-math} % this also loads fontspec
  \defaultfontfeatures{Scale=MatchLowercase}
  \defaultfontfeatures[\rmfamily]{Ligatures=TeX,Scale=1}
\fi
\usepackage{lmodern}
\ifPDFTeX\else
  % xetex/luatex font selection
\fi
% Use upquote if available, for straight quotes in verbatim environments
\IfFileExists{upquote.sty}{\usepackage{upquote}}{}
\IfFileExists{microtype.sty}{% use microtype if available
  \usepackage[]{microtype}
  \UseMicrotypeSet[protrusion]{basicmath} % disable protrusion for tt fonts
}{}
\makeatletter
\@ifundefined{KOMAClassName}{% if non-KOMA class
  \IfFileExists{parskip.sty}{%
    \usepackage{parskip}
  }{% else
    \setlength{\parindent}{0pt}
    \setlength{\parskip}{6pt plus 2pt minus 1pt}}
}{% if KOMA class
  \KOMAoptions{parskip=half}}
\makeatother
\usepackage{xcolor}
\usepackage[margin=1in]{geometry}
\usepackage{color}
\usepackage{fancyvrb}
\newcommand{\VerbBar}{|}
\newcommand{\VERB}{\Verb[commandchars=\\\{\}]}
\DefineVerbatimEnvironment{Highlighting}{Verbatim}{commandchars=\\\{\}}
% Add ',fontsize=\small' for more characters per line
\usepackage{framed}
\definecolor{shadecolor}{RGB}{248,248,248}
\newenvironment{Shaded}{\begin{snugshade}}{\end{snugshade}}
\newcommand{\AlertTok}[1]{\textcolor[rgb]{0.94,0.16,0.16}{#1}}
\newcommand{\AnnotationTok}[1]{\textcolor[rgb]{0.56,0.35,0.01}{\textbf{\textit{#1}}}}
\newcommand{\AttributeTok}[1]{\textcolor[rgb]{0.13,0.29,0.53}{#1}}
\newcommand{\BaseNTok}[1]{\textcolor[rgb]{0.00,0.00,0.81}{#1}}
\newcommand{\BuiltInTok}[1]{#1}
\newcommand{\CharTok}[1]{\textcolor[rgb]{0.31,0.60,0.02}{#1}}
\newcommand{\CommentTok}[1]{\textcolor[rgb]{0.56,0.35,0.01}{\textit{#1}}}
\newcommand{\CommentVarTok}[1]{\textcolor[rgb]{0.56,0.35,0.01}{\textbf{\textit{#1}}}}
\newcommand{\ConstantTok}[1]{\textcolor[rgb]{0.56,0.35,0.01}{#1}}
\newcommand{\ControlFlowTok}[1]{\textcolor[rgb]{0.13,0.29,0.53}{\textbf{#1}}}
\newcommand{\DataTypeTok}[1]{\textcolor[rgb]{0.13,0.29,0.53}{#1}}
\newcommand{\DecValTok}[1]{\textcolor[rgb]{0.00,0.00,0.81}{#1}}
\newcommand{\DocumentationTok}[1]{\textcolor[rgb]{0.56,0.35,0.01}{\textbf{\textit{#1}}}}
\newcommand{\ErrorTok}[1]{\textcolor[rgb]{0.64,0.00,0.00}{\textbf{#1}}}
\newcommand{\ExtensionTok}[1]{#1}
\newcommand{\FloatTok}[1]{\textcolor[rgb]{0.00,0.00,0.81}{#1}}
\newcommand{\FunctionTok}[1]{\textcolor[rgb]{0.13,0.29,0.53}{\textbf{#1}}}
\newcommand{\ImportTok}[1]{#1}
\newcommand{\InformationTok}[1]{\textcolor[rgb]{0.56,0.35,0.01}{\textbf{\textit{#1}}}}
\newcommand{\KeywordTok}[1]{\textcolor[rgb]{0.13,0.29,0.53}{\textbf{#1}}}
\newcommand{\NormalTok}[1]{#1}
\newcommand{\OperatorTok}[1]{\textcolor[rgb]{0.81,0.36,0.00}{\textbf{#1}}}
\newcommand{\OtherTok}[1]{\textcolor[rgb]{0.56,0.35,0.01}{#1}}
\newcommand{\PreprocessorTok}[1]{\textcolor[rgb]{0.56,0.35,0.01}{\textit{#1}}}
\newcommand{\RegionMarkerTok}[1]{#1}
\newcommand{\SpecialCharTok}[1]{\textcolor[rgb]{0.81,0.36,0.00}{\textbf{#1}}}
\newcommand{\SpecialStringTok}[1]{\textcolor[rgb]{0.31,0.60,0.02}{#1}}
\newcommand{\StringTok}[1]{\textcolor[rgb]{0.31,0.60,0.02}{#1}}
\newcommand{\VariableTok}[1]{\textcolor[rgb]{0.00,0.00,0.00}{#1}}
\newcommand{\VerbatimStringTok}[1]{\textcolor[rgb]{0.31,0.60,0.02}{#1}}
\newcommand{\WarningTok}[1]{\textcolor[rgb]{0.56,0.35,0.01}{\textbf{\textit{#1}}}}
\usepackage{graphicx}
\makeatletter
\def\maxwidth{\ifdim\Gin@nat@width>\linewidth\linewidth\else\Gin@nat@width\fi}
\def\maxheight{\ifdim\Gin@nat@height>\textheight\textheight\else\Gin@nat@height\fi}
\makeatother
% Scale images if necessary, so that they will not overflow the page
% margins by default, and it is still possible to overwrite the defaults
% using explicit options in \includegraphics[width, height, ...]{}
\setkeys{Gin}{width=\maxwidth,height=\maxheight,keepaspectratio}
% Set default figure placement to htbp
\makeatletter
\def\fps@figure{htbp}
\makeatother
\setlength{\emergencystretch}{3em} % prevent overfull lines
\providecommand{\tightlist}{%
  \setlength{\itemsep}{0pt}\setlength{\parskip}{0pt}}
\setcounter{secnumdepth}{-\maxdimen} % remove section numbering
\ifLuaTeX
  \usepackage{selnolig}  % disable illegal ligatures
\fi
\usepackage{bookmark}
\IfFileExists{xurl.sty}{\usepackage{xurl}}{} % add URL line breaks if available
\urlstyle{same}
\hypersetup{
  pdftitle={CS 699 Assignment 1},
  pdfauthor={Katherine Rein},
  hidelinks,
  pdfcreator={LaTeX via pandoc}}

\title{CS 699 Assignment 1}
\author{Katherine Rein}
\date{}

\begin{document}
\maketitle

\begin{Shaded}
\begin{Highlighting}[]
\CommentTok{\# Import libraries}
\FunctionTok{library}\NormalTok{(glue)}
\end{Highlighting}
\end{Shaded}

\begin{verbatim}
## Warning: package 'glue' was built under R version 4.3.3
\end{verbatim}

\begin{Shaded}
\begin{Highlighting}[]
\FunctionTok{library}\NormalTok{(modeest)}
\FunctionTok{library}\NormalTok{(dplyr)}
\end{Highlighting}
\end{Shaded}

\begin{verbatim}
## 
## Attaching package: 'dplyr'
\end{verbatim}

\begin{verbatim}
## The following objects are masked from 'package:stats':
## 
##     filter, lag
\end{verbatim}

\begin{verbatim}
## The following objects are masked from 'package:base':
## 
##     intersect, setdiff, setequal, union
\end{verbatim}

\begin{Shaded}
\begin{Highlighting}[]
\FunctionTok{library}\NormalTok{(fastDummies)}
\end{Highlighting}
\end{Shaded}

\begin{verbatim}
## Warning: package 'fastDummies' was built under R version 4.3.3
\end{verbatim}

\begin{Shaded}
\begin{Highlighting}[]
\FunctionTok{library}\NormalTok{(ggplot2)}
\end{Highlighting}
\end{Shaded}

\begin{verbatim}
## Warning: package 'ggplot2' was built under R version 4.3.3
\end{verbatim}

\section{Problem 1}\label{problem-1}

\subsection{Question 1}\label{question-1}

Calculate the mean, median, and standard deviation (sample) of the age
feature

\begin{Shaded}
\begin{Highlighting}[]
\CommentTok{\# Read in data}
\NormalTok{autism\_data }\OtherTok{=} \FunctionTok{read.csv}\NormalTok{(}\StringTok{\textquotesingle{}autism{-}adult.csv\textquotesingle{}}\NormalTok{)}

\CommentTok{\# Drop fully NA rows}
\NormalTok{autism\_data }\OtherTok{\textless{}{-}}\NormalTok{ autism\_data[}\SpecialCharTok{!}\FunctionTok{apply}\NormalTok{(}\FunctionTok{is.na}\NormalTok{(autism\_data), }\DecValTok{1}\NormalTok{, all), ]}

\CommentTok{\# Remove 383 year old}
\NormalTok{autism\_data }\OtherTok{=}\NormalTok{ autism\_data[autism\_data}\SpecialCharTok{$}\NormalTok{age }\SpecialCharTok{!=} \DecValTok{383}\NormalTok{, ]}

\CommentTok{\# Convert age to integer}
\NormalTok{autism\_data}\SpecialCharTok{$}\NormalTok{age }\OtherTok{=} \FunctionTok{as.integer}\NormalTok{(autism\_data}\SpecialCharTok{$}\NormalTok{age)}
\end{Highlighting}
\end{Shaded}

\begin{verbatim}
## Warning: NAs introduced by coercion
\end{verbatim}

\begin{Shaded}
\begin{Highlighting}[]
\CommentTok{\# Calculate the mean median and standard deviation for the age feature}
\NormalTok{mean }\OtherTok{=}  \FunctionTok{mean}\NormalTok{(autism\_data}\SpecialCharTok{$}\NormalTok{age, }\AttributeTok{na.rm =} \ConstantTok{TRUE}\NormalTok{)}
\NormalTok{median }\OtherTok{=} \FunctionTok{median}\NormalTok{(autism\_data}\SpecialCharTok{$}\NormalTok{age, }\AttributeTok{na.rm =} \ConstantTok{TRUE}\NormalTok{)}
\NormalTok{stdev }\OtherTok{=} \FunctionTok{sd}\NormalTok{(autism\_data}\SpecialCharTok{$}\NormalTok{age, }\AttributeTok{na.rm =} \ConstantTok{TRUE}\NormalTok{)}

\CommentTok{\# Print answers}
\FunctionTok{glue}\NormalTok{(}\StringTok{\textquotesingle{}Mean: \{mean\}\textquotesingle{}}\NormalTok{)}
\end{Highlighting}
\end{Shaded}

\begin{verbatim}
## Mean: 29.1940085592011
\end{verbatim}

\begin{Shaded}
\begin{Highlighting}[]
\FunctionTok{glue}\NormalTok{(}\StringTok{\textquotesingle{}Median: \{median\}\textquotesingle{}}\NormalTok{)}
\end{Highlighting}
\end{Shaded}

\begin{verbatim}
## Median: 27
\end{verbatim}

\begin{Shaded}
\begin{Highlighting}[]
\FunctionTok{glue}\NormalTok{(}\StringTok{\textquotesingle{}Standard Deviation (sample): \{stdev\}\textquotesingle{}}\NormalTok{)}
\end{Highlighting}
\end{Shaded}

\begin{verbatim}
## Standard Deviation (sample): 9.71152590893556
\end{verbatim}

\begin{verbatim}
After some preliminary data analysis, I noticed there was someone who was 383 years old
which seems impossible. I removed this entry before continuing on.
The mean of the data set is 29.20. The median is 27. The sample standard deviation is 9.71.
\end{verbatim}

\subsection{Question 2}\label{question-2}

Determine Q1, Q2, and Q3 of age

\begin{Shaded}
\begin{Highlighting}[]
\CommentTok{\# Calculate Q1, Q2, Q3}
\NormalTok{quantile\_vector }\OtherTok{=} \FunctionTok{quantile}\NormalTok{(autism\_data}\SpecialCharTok{$}\NormalTok{age, }\AttributeTok{probs =} \FunctionTok{c}\NormalTok{(}\FloatTok{0.25}\NormalTok{, }\FloatTok{0.5}\NormalTok{, }\FloatTok{0.75}\NormalTok{), }\AttributeTok{na.rm =} \ConstantTok{TRUE}\NormalTok{)}

\CommentTok{\# Store individually}
\NormalTok{Q1 }\OtherTok{=}\NormalTok{ quantile\_vector[}\DecValTok{1}\NormalTok{] }
\NormalTok{Q2 }\OtherTok{=}\NormalTok{ quantile\_vector[}\DecValTok{2}\NormalTok{] }
\NormalTok{Q3 }\OtherTok{=}\NormalTok{ quantile\_vector[}\DecValTok{3}\NormalTok{]}

\CommentTok{\# Print}
\FunctionTok{glue}\NormalTok{(}\StringTok{\textquotesingle{}Q1: \{Q1\}, Q2: \{Q2\}, Q3: \{Q3\}\textquotesingle{}}\NormalTok{)}
\end{Highlighting}
\end{Shaded}

\begin{verbatim}
## Q1: 21, Q2: 27, Q3: 35
\end{verbatim}

\begin{verbatim}
Q2 is the same as the median of the data set but I also recalculated it. Q1 is 21, Q2 is 27, and Q3 is 35.
\end{verbatim}

\subsection{Question 3}\label{question-3}

Plot the boxplot of the age feature

\begin{Shaded}
\begin{Highlighting}[]
\CommentTok{\# Plot Boxplot}
\FunctionTok{boxplot}\NormalTok{(autism\_data}\SpecialCharTok{$}\NormalTok{age, }
        \AttributeTok{main =} \StringTok{"Distribution of Age"}\NormalTok{, }\AttributeTok{ylab =} \StringTok{"Age"}\NormalTok{,}
        \AttributeTok{col =} \StringTok{"lightblue"}\NormalTok{, }\AttributeTok{notch =} \ConstantTok{FALSE}\NormalTok{, }\AttributeTok{na.rm =} \ConstantTok{TRUE}\NormalTok{)}
\end{Highlighting}
\end{Shaded}

\includegraphics{CS699_Assignment1_files/figure-latex/unnamed-chunk-4-1.pdf}

\subsection{Question 4}\label{question-4}

Implement min-max rescaling on the age feature. Replace the original age
feature with the rescaled result. Provide the rescaled age for the
seventh observation in the data.

\begin{Shaded}
\begin{Highlighting}[]
\CommentTok{\# Find the minimum and maximum age}
\NormalTok{min\_age }\OtherTok{=} \FunctionTok{min}\NormalTok{(autism\_data}\SpecialCharTok{$}\NormalTok{age, }\AttributeTok{na.rm =} \ConstantTok{TRUE}\NormalTok{) }
\NormalTok{max\_age }\OtherTok{=} \FunctionTok{max}\NormalTok{(autism\_data}\SpecialCharTok{$}\NormalTok{age, }\AttributeTok{na.rm =} \ConstantTok{TRUE}\NormalTok{)}

\CommentTok{\# Implement min‐max rescaling to the [0, 1] interval}
\CommentTok{\# Formula: (x − min) / (max − min)}
\NormalTok{autism\_data}\SpecialCharTok{$}\NormalTok{age }\OtherTok{=}\NormalTok{ (autism\_data}\SpecialCharTok{$}\NormalTok{age }\SpecialCharTok{{-}}\NormalTok{ min\_age) }\SpecialCharTok{/}\NormalTok{ (max\_age }\SpecialCharTok{{-}}\NormalTok{ min\_age)}

\CommentTok{\# Print seventh observation}
\NormalTok{seventh\_obs }\OtherTok{=}\NormalTok{ autism\_data}\SpecialCharTok{$}\NormalTok{age[}\DecValTok{7}\NormalTok{]}
\FunctionTok{print}\NormalTok{(seventh\_obs)}
\end{Highlighting}
\end{Shaded}

\begin{verbatim}
## [1] 0
\end{verbatim}

\begin{verbatim}
For min-max rescaling I used a minimum value of 0 and max of 1. The seventh observation
is now 0. This means that before the data was rescaled it was the minimum age.
\end{verbatim}

\subsection{Question 5}\label{question-5}

Determine the mode of the country\_of\_res feature

\begin{Shaded}
\begin{Highlighting}[]
\CommentTok{\# Handle NAs}
\NormalTok{autism\_data}\SpecialCharTok{$}\NormalTok{country\_of\_res }\OtherTok{\textless{}{-}}\NormalTok{ dplyr}\SpecialCharTok{::}\FunctionTok{na\_if}\NormalTok{(autism\_data}\SpecialCharTok{$}\NormalTok{country\_of\_res, }\StringTok{\textquotesingle{}\textquotesingle{}}\NormalTok{)}

\CommentTok{\# Create a frequency table of unique values}
\NormalTok{freq\_loc }\OtherTok{\textless{}{-}} \FunctionTok{table}\NormalTok{(autism\_data}\SpecialCharTok{$}\NormalTok{country\_of\_res, }\AttributeTok{useNA =} \StringTok{"no"}\NormalTok{)}

\CommentTok{\# Find the most frequent values of a vector}
\NormalTok{mode\_loc }\OtherTok{\textless{}{-}}\NormalTok{ modeest}\SpecialCharTok{::}\FunctionTok{mfv}\NormalTok{(autism\_data}\SpecialCharTok{$}\NormalTok{country\_of\_res, }\AttributeTok{na\_rm =} \ConstantTok{TRUE}\NormalTok{)}

\CommentTok{\# Concatenate frequencies and modes}
\FunctionTok{list}\NormalTok{(}\AttributeTok{frequencies =}\NormalTok{ freq\_loc, }\AttributeTok{mode =}\NormalTok{ mode\_loc)}
\end{Highlighting}
\end{Shaded}

\begin{verbatim}
## $frequencies
## 
##           'Costa Rica'       'Czech Republic'            'Hong Kong' 
##                      1                      1                      1 
##          'New Zealand'         'Saudi Arabia'         'Sierra Leone' 
##                     80                      4                      1 
##         'South Africa'            'Sri Lanka' 'United Arab Emirates' 
##                      2                     14                     82 
##       'United Kingdom'        'United States'             'Viet Nam' 
##                     77                    113                      5 
##            Afghanistan          AmericanSamoa                 Angola 
##                     13                      2                      1 
##              Argentina                Armenia                  Aruba 
##                      2                      2                      1 
##              Australia                Austria             Azerbaijan 
##                     27                      4                      1 
##                Bahamas             Bangladesh                Belgium 
##                      2                      3                      3 
##                Bolivia                 Brazil                Burundi 
##                      1                      9                      1 
##                 Canada                  Chile                  China 
##                     15                      1                      1 
##                 Cyprus                Ecuador                  Egypt 
##                      1                      1                      3 
##               Ethiopia                Finland                 France 
##                      2                      1                     11 
##                Germany                Iceland                  India 
##                      4                      2                     81 
##              Indonesia                   Iran                   Iraq 
##                      1                      7                      1 
##                Ireland                  Italy                  Japan 
##                      5                      5                      1 
##                 Jordan             Kazakhstan                Lebanon 
##                     47                      3                      1 
##               Malaysia                 Mexico                  Nepal 
##                      5                      8                      1 
##            Netherlands              Nicaragua                  Niger 
##                     10                      1                      1 
##                   Oman               Pakistan            Philippines 
##                      1                      3                      4 
##               Portugal                Romania                 Russia 
##                      1                      3                      7 
##                 Serbia                  Spain                 Sweden 
##                      1                      3                      2 
##                  Tonga                 Turkey                Ukraine 
##                      1                      1                      2 
##                Uruguay 
##                      1 
## 
## $mode
## [1] "'United States'"
\end{verbatim}

\begin{verbatim}
The mode of the country of residence column is United States with 113 entries.
\end{verbatim}

\subsection{Question 6}\label{question-6}

Review the ethnicity feature. You will notice several missing values in
this feature. Determine a reasonable imputation for this feature.
Explain what you are going to do and why. Then replace the original
ethnicity feature with the imputed result.

\begin{Shaded}
\begin{Highlighting}[]
\CommentTok{\# Identify all categories used in the ethnicity feature}
\FunctionTok{unique}\NormalTok{(autism\_data}\SpecialCharTok{$}\NormalTok{ethnicity)}
\end{Highlighting}
\end{Shaded}

\begin{verbatim}
##  [1] "White-European"    "Latino"            "?"                
##  [4] "Others"            "Black"             "Asian"            
##  [7] "'Middle Eastern '" "Pasifika"          "'South Asian'"    
## [10] "Hispanic"          "Turkish"           "others"
\end{verbatim}

\begin{Shaded}
\begin{Highlighting}[]
\CommentTok{\# Turn missing values into NA}
\NormalTok{autism\_data}\SpecialCharTok{$}\NormalTok{ethnicity }\OtherTok{\textless{}{-}}\NormalTok{ dplyr}\SpecialCharTok{::}\FunctionTok{na\_if}\NormalTok{(autism\_data}\SpecialCharTok{$}\NormalTok{ethnicity, }\StringTok{\textquotesingle{}\textquotesingle{}}\NormalTok{)}

\CommentTok{\# Change all other values into one other value}
\NormalTok{autism\_data }\OtherTok{\textless{}{-}}\NormalTok{ autism\_data }\SpecialCharTok{\%\textgreater{}\%}
  \FunctionTok{mutate}\NormalTok{(}\AttributeTok{ethnicity =} \FunctionTok{case\_when}\NormalTok{(}
\NormalTok{    ethnicity }\SpecialCharTok{==} \StringTok{\textquotesingle{}?\textquotesingle{}} \SpecialCharTok{\textasciitilde{}} \StringTok{\textquotesingle{}Other\textquotesingle{}}\NormalTok{,}
\NormalTok{    ethnicity }\SpecialCharTok{==} \StringTok{\textquotesingle{}others\textquotesingle{}} \SpecialCharTok{\textasciitilde{}} \StringTok{\textquotesingle{}Other\textquotesingle{}}\NormalTok{,}
\NormalTok{    ethnicity }\SpecialCharTok{==} \StringTok{\textquotesingle{}Others\textquotesingle{}} \SpecialCharTok{\textasciitilde{}} \StringTok{\textquotesingle{}Other\textquotesingle{}}\NormalTok{,}
\NormalTok{    ethnicity }\SpecialCharTok{==} \ConstantTok{NA} \SpecialCharTok{\textasciitilde{}} \StringTok{\textquotesingle{}Other\textquotesingle{}}\NormalTok{,}
    \ConstantTok{TRUE} \SpecialCharTok{\textasciitilde{}}\NormalTok{ ethnicity}
\NormalTok{  ))}
\end{Highlighting}
\end{Shaded}

\begin{verbatim}
After looking at all of the unique values for the ethnicity feature, I noticed there were many entries that meant other. I then changed all of them to other. This seemed like the best way to not overinflate one category (as we don't know where these individuals came from). 
\end{verbatim}

\subsection{Question 7}\label{question-7}

Create a bar graph of your imputed ethnicity feature.

\begin{Shaded}
\begin{Highlighting}[]
\FunctionTok{barplot}\NormalTok{(}\FunctionTok{table}\NormalTok{(autism\_data}\SpecialCharTok{$}\NormalTok{ethnicity),}
        \AttributeTok{main =} \StringTok{"Ethnicity Distribution"}\NormalTok{,}
        \AttributeTok{xlab =} \StringTok{"Ethnicity"}\NormalTok{,}
        \AttributeTok{ylab =} \StringTok{"Count"}\NormalTok{,}
        \AttributeTok{col =} \StringTok{"skyblue"}\NormalTok{,}
        \AttributeTok{cex.names =} \FloatTok{0.6}\NormalTok{,}
        \AttributeTok{las =} \DecValTok{2}\NormalTok{)}
\end{Highlighting}
\end{Shaded}

\includegraphics{CS699_Assignment1_files/figure-latex/unnamed-chunk-8-1.pdf}

\subsection{Question 8}\label{question-8}

Implement dummy coding for the gender feature. Replace the original
gender feature with the coded result. Provide the coded gender for the
last ten observations in the data.

\begin{Shaded}
\begin{Highlighting}[]
\CommentTok{\# Ensure gender is treated as a categorical factor}
\NormalTok{autism\_data}\SpecialCharTok{$}\NormalTok{gender }\OtherTok{\textless{}{-}} \FunctionTok{as.factor}\NormalTok{(autism\_data}\SpecialCharTok{$}\NormalTok{gender)}

\CommentTok{\# One encoding of gender column}
\NormalTok{d }\OtherTok{\textless{}{-}}\NormalTok{ fastDummies}\SpecialCharTok{::}\FunctionTok{dummy\_cols}\NormalTok{(autism\_data,}
                             \AttributeTok{select\_columns =} \StringTok{"gender"}\NormalTok{, }
                             \AttributeTok{remove\_selected\_columns =} \ConstantTok{TRUE}\NormalTok{, }
                             \AttributeTok{remove\_first\_dummy =} \ConstantTok{TRUE}
\NormalTok{                             )}

\CommentTok{\# Show the last 10}
\NormalTok{coded\_gender\_last10 }\OtherTok{=} \FunctionTok{tail}\NormalTok{(d[ , }\FunctionTok{grep}\NormalTok{(}\StringTok{"\^{}gender\_"}\NormalTok{, }\FunctionTok{names}\NormalTok{(d)) ], }\DecValTok{10}\NormalTok{)}
\NormalTok{coded\_gender\_last10}
\end{Highlighting}
\end{Shaded}

\begin{verbatim}
##  [1] 1 1 1 0 0 0 1 0 1 0
\end{verbatim}

\subsection{Question 9}\label{question-9}

Identify which features in your data set are discrete and which are
continuous.

\begin{verbatim}
From visual investigation, it seems that the following features are discrete: A1_score
- A10_score, gender, ethnicity, jaundice, autism, country of residence, used app before,
relation, and class ASD. The only continuous feature is age.
\end{verbatim}

\subsection{Question 10}\label{question-10}

Identify which features in your data set are numeric and which are
non-numeric. Compare with the discrete/continuous classification you
just made and discuss the similarities and/or differences you see.

\begin{Shaded}
\begin{Highlighting}[]
\CommentTok{\# Convert blank strings to NA}
\NormalTok{autism\_data }\OtherTok{\textless{}{-}} \FunctionTok{mutate}\NormalTok{(autism\_data, }\FunctionTok{across}\NormalTok{(}\FunctionTok{where}\NormalTok{(is.character), }\SpecialCharTok{\textasciitilde{}}\NormalTok{ dplyr}\SpecialCharTok{::}\FunctionTok{na\_if}\NormalTok{(.x, }\StringTok{\textquotesingle{}\textquotesingle{}}\NormalTok{)))}

\CommentTok{\# Identify column classes of the data}
\NormalTok{col\_classes }\OtherTok{\textless{}{-}} \FunctionTok{sapply}\NormalTok{(autism\_data, class)}

\CommentTok{\# Link classes and names}
\NormalTok{col\_classes\_df }\OtherTok{=} \FunctionTok{data.frame}\NormalTok{(}\AttributeTok{Column =} \FunctionTok{names}\NormalTok{(autism\_data), }\AttributeTok{Class =}\NormalTok{ col\_classes)}

\FunctionTok{print}\NormalTok{(col\_classes\_df)}
\end{Highlighting}
\end{Shaded}

\begin{verbatim}
##                          Column     Class
## A1_Score               A1_Score   integer
## A2_Score               A2_Score   integer
## A3_Score               A3_Score   integer
## A4_Score               A4_Score   integer
## A5_Score               A5_Score   integer
## A6_Score               A6_Score   integer
## A7_Score               A7_Score   integer
## A8_Score               A8_Score   integer
## A9_Score               A9_Score   integer
## A10_Score             A10_Score   integer
## age                         age   numeric
## gender                   gender    factor
## ethnicity             ethnicity character
## jaundice               jaundice character
## austim                   austim character
## country_of_res   country_of_res character
## used_app_before used_app_before character
## relation               relation character
## Class.ASD             Class.ASD character
\end{verbatim}

\begin{verbatim}
The following columns are numeric: A1_Score - A10_Score and age. The non numeric columns
are: gender, ethnicity, jaundice, autism, country of residence, used app before,
relation, and Class ASD. The only difference between numeric and continuous is that
the A score columns are discrete data with numeric classes. This makes them a numeric
discrete feature which is unlike any other column.
\end{verbatim}

\subsection{Question 11}\label{question-11}

After completing all requested tasks above, print the first 4
observations of the data.

\begin{Shaded}
\begin{Highlighting}[]
\FunctionTok{head}\NormalTok{(autism\_data, }\AttributeTok{n =} \DecValTok{4}\NormalTok{)}
\end{Highlighting}
\end{Shaded}

\begin{verbatim}
##   A1_Score A2_Score A3_Score A4_Score A5_Score A6_Score A7_Score A8_Score
## 1        1        1        1        1        0        0        1        1
## 2        1        1        0        1        0        0        0        1
## 3        1        1        0        1        1        0        1        1
## 4        1        1        0        1        0        0        1        1
##   A9_Score A10_Score       age gender      ethnicity jaundice austim
## 1        0         0 0.1914894      f White-European       no     no
## 2        0         1 0.1489362      m         Latino       no    yes
## 3        1         1 0.2127660      m         Latino      yes    yes
## 4        0         1 0.3829787      f White-European       no    yes
##    country_of_res used_app_before relation Class.ASD
## 1 'United States'              no     Self        NO
## 2          Brazil              no     Self        NO
## 3           Spain              no   Parent       YES
## 4 'United States'              no     Self        NO
\end{verbatim}

\section{Problem 2}\label{problem-2}

\subsection{Question 1}\label{question-1-1}

Create a scatterplot of feature A1 vs.~feature A5.

\begin{Shaded}
\begin{Highlighting}[]
\CommentTok{\# Read in data}
\NormalTok{corr\_data }\OtherTok{=} \FunctionTok{read.csv}\NormalTok{(}\StringTok{\textquotesingle{}correlation.csv\textquotesingle{}}\NormalTok{)}

\CommentTok{\# Drop NAs in both columns}
\NormalTok{corr\_clean }\OtherTok{\textless{}{-}}\NormalTok{ tidyr}\SpecialCharTok{::}\FunctionTok{drop\_na}\NormalTok{(corr\_data, A1, A5)}

\CommentTok{\# Make all values numeric}
\NormalTok{corr\_clean}\SpecialCharTok{$}\NormalTok{A1 }\OtherTok{\textless{}{-}} \FunctionTok{suppressWarnings}\NormalTok{(}\FunctionTok{as.numeric}\NormalTok{(corr\_clean}\SpecialCharTok{$}\NormalTok{A1))}
\NormalTok{corr\_clean}\SpecialCharTok{$}\NormalTok{A5 }\OtherTok{\textless{}{-}} \FunctionTok{suppressWarnings}\NormalTok{(}\FunctionTok{as.numeric}\NormalTok{(corr\_clean}\SpecialCharTok{$}\NormalTok{A5))}

\CommentTok{\# Create scatterplot}
\FunctionTok{ggplot}\NormalTok{(corr\_clean, }\FunctionTok{aes}\NormalTok{(}\AttributeTok{x =}\NormalTok{ A1, }\AttributeTok{y =}\NormalTok{ A5)) }\SpecialCharTok{+}
  \FunctionTok{geom\_point}\NormalTok{(}\AttributeTok{color =} \StringTok{"steelblue"}\NormalTok{) }\SpecialCharTok{+}
  \FunctionTok{geom\_smooth}\NormalTok{(}\AttributeTok{method =} \StringTok{"lm"}\NormalTok{, }\AttributeTok{se =} \ConstantTok{FALSE}\NormalTok{, }\AttributeTok{linewidth =} \FloatTok{0.8}\NormalTok{) }\SpecialCharTok{+} \FunctionTok{labs}\NormalTok{(}\AttributeTok{title =} \StringTok{"A1 vs. A5 Scatterplot"}\NormalTok{,}
                                                                 \AttributeTok{x =} \StringTok{"Feature A1"}\NormalTok{,}
                                                                 \AttributeTok{y =} \StringTok{"Feature A5"}\NormalTok{) }\SpecialCharTok{+} \FunctionTok{theme\_minimal}\NormalTok{()}
\end{Highlighting}
\end{Shaded}

\begin{verbatim}
## `geom_smooth()` using formula = 'y ~ x'
\end{verbatim}

\includegraphics{CS699_Assignment1_files/figure-latex/unnamed-chunk-12-1.pdf}

\subsection{Question 2}\label{question-2-1}

Compute the correlation matrix for all five features in the data set.

\begin{Shaded}
\begin{Highlighting}[]
\CommentTok{\# Create a vector of column names}
\NormalTok{cols }\OtherTok{=} \FunctionTok{paste0}\NormalTok{(}\StringTok{"A"}\NormalTok{, }\DecValTok{1}\SpecialCharTok{:}\DecValTok{5}\NormalTok{)}

\CommentTok{\# Ensure all columns are numeric}
\ControlFlowTok{for}\NormalTok{ (v }\ControlFlowTok{in}\NormalTok{ cols) corr\_data[[v]] }\OtherTok{\textless{}{-}} \FunctionTok{suppressWarnings}\NormalTok{(}\FunctionTok{as.numeric}\NormalTok{(corr\_data[[v]]))}

\CommentTok{\# Compute the Pearson correlation matrix}
\NormalTok{cor\_mat }\OtherTok{\textless{}{-}} \FunctionTok{cor}\NormalTok{(corr\_data[cols], }\AttributeTok{use =} \StringTok{"pairwise.complete.obs"}\NormalTok{, }\AttributeTok{method =} \StringTok{"pearson"}\NormalTok{)}

\CommentTok{\# Print correlation matrix}
\NormalTok{cor\_mat}
\end{Highlighting}
\end{Shaded}

\begin{verbatim}
##           A1         A2        A3         A4         A5
## A1 1.0000000 0.17880899 0.2896419 0.17717981 0.45241040
## A2 0.1788090 1.00000000 0.4645427 0.34013541 0.04235286
## A3 0.2896419 0.46454268 1.0000000 0.29379683 0.19451370
## A4 0.1771798 0.34013541 0.2937968 1.00000000 0.06186254
## A5 0.4524104 0.04235286 0.1945137 0.06186254 1.00000000
\end{verbatim}

\subsection{Question 3}\label{question-3-1}

Identify the strongest correlation in the data set. Which factors are
involved? Is it a positive correlation or a negative correlation?

\begin{Shaded}
\begin{Highlighting}[]
\CommentTok{\# Switch diagonal to NA}
\FunctionTok{diag}\NormalTok{(cor\_mat) }\OtherTok{\textless{}{-}} \ConstantTok{NA}

\CommentTok{\# Find the max correlation value}
\NormalTok{max\_val }\OtherTok{=} \FunctionTok{max}\NormalTok{(cor\_mat, }\AttributeTok{na.rm =} \ConstantTok{TRUE}\NormalTok{)}

\CommentTok{\# Find the position of that max value}
\FunctionTok{which}\NormalTok{(cor\_mat }\SpecialCharTok{==}\NormalTok{ max\_val, }\AttributeTok{arr.ind =} \ConstantTok{TRUE}\NormalTok{)}
\end{Highlighting}
\end{Shaded}

\begin{verbatim}
##    row col
## A3   3   2
## A2   2   3
\end{verbatim}

\begin{verbatim}
The strongest correlation is 0.465 which correlates A3 and A2. This is a positive 
correlation because the number is positive.
\end{verbatim}

\subsection{Question 4}\label{question-4-1}

Implement z-score normalization on all features in the data set

\begin{Shaded}
\begin{Highlighting}[]
\NormalTok{corr\_data[cols] }\OtherTok{\textless{}{-}} \FunctionTok{as.data.frame}\NormalTok{(}\FunctionTok{scale}\NormalTok{(corr\_data[cols], }\AttributeTok{center =} \ConstantTok{TRUE}\NormalTok{, }\AttributeTok{scale =} \ConstantTok{TRUE}\NormalTok{))}
\end{Highlighting}
\end{Shaded}

\subsection{Question 5}\label{question-5-1}

Compute the correlation matrix for all five normalized features in the
data set. Compare this correlation matrix with the matrix you obtained
earlier and discuss the similarities and/or differences you see.

\begin{Shaded}
\begin{Highlighting}[]
\CommentTok{\# Create a vector of column names}
\NormalTok{cols }\OtherTok{=} \FunctionTok{paste0}\NormalTok{(}\StringTok{"A"}\NormalTok{, }\DecValTok{1}\SpecialCharTok{:}\DecValTok{5}\NormalTok{)}

\CommentTok{\# Ensure all columns are numeric}
\ControlFlowTok{for}\NormalTok{ (v }\ControlFlowTok{in}\NormalTok{ cols) corr\_data[[v]] }\OtherTok{\textless{}{-}} \FunctionTok{suppressWarnings}\NormalTok{(}\FunctionTok{as.numeric}\NormalTok{(corr\_data[[v]]))}

\CommentTok{\# Compute the Pearson correlation matrix}
\NormalTok{cor\_mat }\OtherTok{\textless{}{-}} \FunctionTok{cor}\NormalTok{(corr\_data[cols], }\AttributeTok{use =} \StringTok{"pairwise.complete.obs"}\NormalTok{, }\AttributeTok{method =} \StringTok{"pearson"}\NormalTok{)}

\CommentTok{\# Print correlation matrix}
\NormalTok{cor\_mat}
\end{Highlighting}
\end{Shaded}

\begin{verbatim}
##           A1         A2        A3         A4         A5
## A1 1.0000000 0.17880899 0.2896419 0.17717981 0.45241040
## A2 0.1788090 1.00000000 0.4645427 0.34013541 0.04235286
## A3 0.2896419 0.46454268 1.0000000 0.29379683 0.19451370
## A4 0.1771798 0.34013541 0.2937968 1.00000000 0.06186254
## A5 0.4524104 0.04235286 0.1945137 0.06186254 1.00000000
\end{verbatim}

\begin{verbatim}
The correlation matricies are identical. This makes sense because when we normalize 
features we are already removing the affects of mean and standard deviation.
\end{verbatim}

\end{document}
